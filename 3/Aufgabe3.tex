\documentclass{article}
\usepackage[utf8]{inputenc}
\usepackage[ngerman]{babel}
\usepackage{amsmath}

\begin{document}
\section{Aufgabe 3}
Während man durch die Elemente geht, werden diese nummeriert. Nach 
Ausführung des Sortieralgorithmus prüft man für gleich große Elemente,
ob ihre Nummern auch sortiert sind, und tauscht die Elemente eventuell 
noch.
Im Worst-Case sind alle Elemente gleich groß und werden beim Sortieren
genau umgedreht. Dann muss man genau doppelt so viele Vergleiche  wie in
der ursprünglichen Version anstellen.  
\end{document}
