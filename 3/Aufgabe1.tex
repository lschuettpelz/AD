\documentclass{article}
\usepackage[utf8]{inputenc}
\usepackage[ngerman]{babel}
\usepackage{amsmath}
\usepackage{algorithm}
\usepackage{algpseudocode}

\begin{document}
\section{Aufgabe 1}
\begin{algorithm}
\begin{algorithmic}

\Function{Schleifenprüfer}{Liste L}
\For {$i \gets 0, \dots, Liste.length$}
	\State $\Call{Verkettung}{a, b} \gets \Call{Verkettung}{b,a}$
	\If {$i> 0 \text{ und }  L[i] = L[1]$} \\
	\Return $\text{Enthält Schleife}$
	\Else \\
	\Return $\text{Enthält keine Schleife}$
	\EndIf
\EndFor
\EndFunction
\end{algorithmic}
\end{algorithm}
Laut Aufgabe soll mit dem Algorithmus herausgefunden werden, ob eine 
gegebene verkettete Liste eine Schleife enthält oder nicht (also das 
letzte Element auf Null zeigt.) Dazu gehen wir durch die Liste durch 
und drehen jede Verkettung um. Wenn eine Schleife drin ist, so wird nun 
beim letzten Element auf ein vorheriges verwiesen, dieses führt aber 
nun wenn wir der Verkettung folgen wieder zum ersten Element. Damit 
muss die Liste im schlimmsten Fall zweimal abgelaufen werden, wenn sie 
danach wieder umgedreht werden soll, viermal. Diese Fälle liegen alle 
in O(n). Es wird kein zusätzlicher Speicher benötgt.
\end{document}
