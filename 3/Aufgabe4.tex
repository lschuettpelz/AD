\documentclass{article}
\usepackage[utf8]{inputenc}
\usepackage[ngerman]{babel}
\usepackage{amsmath}
\usepackage{algorithm}
\usepackage{algpseudocode}
\usepackage{tikz}

\begin{document}
\section{Aufgabe 4}
\begin{enumerate}
\item[(a)]
\begin{center}
\begin{tikzpicture}

\node[]


\end{tikzpicture}
\end{center}
\item[(b)]
Es wird in den Bubblesort-Algorithmus eine boolesche Variabe hinzugefügt,
die zu Beginn auf false gesetzt ist. Wird die innere for-Schleife 
durchlaufen, ohne dass ein Element getauscht wird, so bleibt die 
Variable auf false und man kann erkennen, dass kein Element mehr 
getauscht wurde, also der Array nicht weiter durchlaufen werden muss.
Die best-case Laufzeit kann dann erzielt werden, wenn der eingegebene
Array bereits sortiert ist. Hierbei kann nach n Prüfungen bereits 
abgeschlossen werden, also beträgt der Aufwand O(n).
\item[(c)]
\begin{algorithm}
\begin{algorithmic}
\Function{Bubblesort+}{A}
\For{$i \gets 0, \dots , A.length$}
	\State $b \gets false$
	\For{$j \gets A.length, \dots, i+1$}
		\If{$A[i] < A[i-1]$}
		\Call{Change}{$A[i], A[i-1]$}
		\State $b \gets true$
		\EndIf
	\EndFor
\EndFor
\EndFunction
\end{algorithmic}
\end{algorithm}
\item[(d)]
Der Mehraufwand im worst-case beträgt $O(1)$, da eine Variable gespeichert
werden muss. Damit bleibt die worst-case Laufzeit bei $O(n^2)$, und da 
sich aber die best-case Laufzeit verbessert, lohnt sich die Adaption.
\item[(e)]

Der Baum sieht gleich aus bis auf die best-case Version des schon 
sortierten Arrays, bei der sich der Baum verkürzt.
\end{enumerate}
\end{document}
