\documentclass[article]
\usepackage[ngerma]{babel}
\usepackage[utf8]{inputenc}
\usepackage{algpseudocode}

\begin{document}
\section{Aufgabe 3}
\begin{enumerate}
\item[(a)]
Ein Stack wird dazu genutzt, um neue Elemente zu "enqueuen", wird 
also "enqueue" aufgerufen, so wird betreffendes Element mit "push" 
auf den zweiten Stack gelegt. Der andere Stack wird für den Abruf 
von Elementen genutzt, mit "pop" werden Elemente "dequeued". Ist der 
erste Stack leer, so wird der zweite Stack umgedreht auf den ersten 
gelegt, also das oberste Element des zweiten Stacks wird gepopt, 
auf den ersten Stak gepusht und damit zum untersten Element. Dann 
wird das zweite Element des zweiten Stacks gepopt, gepusht und zum 
vorletzten Element....

\begin{algorithmic}
\State $stack Stack1 \gets new stack()$
\State $stack Stack2 \gets new stack()$

\Function{enqueue}{Element e}
Stack2.push(e);
\EndFunction

\Function{dequeue}{}
{
element1 = Stack1.pop();
if{element1 = null}
\State	element = Stack2.pop();
	while{element != null}
		for{ i=0, i<Stack2.lenght()}
		\State Stack1.push(element);
		element = Stack2.pop()
		
	
else 
{ 
\Return element1}
\EndFunction
\end{algorithmic}
\item[(b)]
\end{enumerate}
\end{document}
