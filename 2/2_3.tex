\documentclass{scrartcl}
\usepackage[ngerman]{babel}
\usepackage[utf8]{inputenc}
\usepackage{algorithm}
\usepackage{algpseudocode}

\begin{document}
\section{Aufgabe 3}
\begin{enumerate}
\item[(a)]
Ein Stack wird dazu genutzt, um neue Elemente zu "`enqueuen"', wird 
also \textit{enqueue} aufgerufen, so wird betreffendes Element mit 
\textit{push} auf den zweiten Stack gelegt. Der andere Stack wird 
für den Abruf von Elementen genutzt, mit \textit{pop} werden Elemente 
"`dequeued"'. Ist der erste Stack leer, so wird der zweite Stack 
umgedreht auf den Ersten gelegt, also das oberste Element des zweiten 
Stacks wird "`gepopt"', auf den ersten Stack "`gepusht"' und damit 
zum untersten Element. Dann wird das zweite Element des zweiten Stacks 
"`gepopt"', "`gepusht"' und zum vorletzten Element....

\begin{algorithm}
\caption{Queue}
\begin{algorithmic}[1]
\State $stack Stack1 \gets new stack()$
\State $stack Stack2 \gets new stack()$

\Function{$enqueue$}{$Element e$} \\
\Call{$Stack2.push$}{$e$}
\EndFunction

\Function{$dequeue$}{} 
\State $element1 \gets \Call{$Stack1.pop$}$
\If{$element1 == null$}
\State	$element2 \gets \Call{$Stack2.pop$}$
	\While{$element2 \neq null$}
		\Call{$Stack1.push$}{$element2$}
		\State $element2 \gets \Call{$Stack2.pop$}$
	\EndWhile
\Else \\
	\Return $element1$
\EndIf
\EndFunction
\end{algorithmic}
\end{algorithm}

Der Aufwand von \textit{enqueue} ist konstant ($O(1)$), doch der Aufwand 
von \textit{dequeue} ist linear abhängig von der Anzahl der Element im 
zweiten Stack ($O(n)$).

\item[(b)]
Der Worst-Case entspricht dem Fall, dass man erst $n-1$ Elemente 
pusht, und dann eines abruft, sodass der gesamte zweite Stack 
umgedreht werden muss. Damit hat man eine Laufzeit $T_n$ von $n-1+1$, 
also $n$. Damit ist die \textit{amortisierte Laufzeit} dieser 
Operationen $T_n/n$ 1. 

(Auch in anderen Fällen ist die Laufzeit der $n_E$ Enqueue Operationen 
+ $n_D$ Dequeue-Operationen genau $O(n)$, da Enqueue nur $n_E \cdot 1$ 
Operationen braucht, und Dequeue nur $b \cdot \frac{n_D}{b}$, also 
$n_D$ Operationen, entweder $n \cdot 1$ Operationen, $1 \cdot n$ 
Operationen oder beliebige Kombinationen dazwischen.)
\end{enumerate}
\end{document}
