\documentclass{scrartcl}
\usepackage[ngerman]{babel}
\usepackage[utf8]{inputenc}

\usepackage{amsmath}
\usepackage{amssymb}
\usepackage{latexsym}

\usepackage{algorithm}
\usepackage{algpseudocode}

\usepackage{tikz}

\begin{document}
\section{Aufgabe 3}
\begin{enumerate}
\item[(a)]
Gegeben sei die Funktion $f : \mathbb{R}_{>1} \to \mathbb{R}$ mit
\[
    f(x) = x \log_n x = \log_e n \cdot \left( x \cdot \frac{1}{\log_e x} \right)
\]
Gesucht sei nun ein Minimum der Funktion $f$. Zunächst bilden wir die
Ableitungen der Funktion:
\begin{align*}
f(x) &= \log_e n \cdot \left( x \cdot \frac{1}{\log_e x} \right) \\
\frac{\mathrm d}{\mathrm d x} f(x) &= \log_e n \cdot \frac{\log_e x - 1}{\log_e^2 x} \\
\frac{\mathrm d^2}{\mathrm d x^2} f(x) &= \log_e n \cdot \frac{2 - \log_e x}{x \cdot \log_e^3 x}
\end{align*}
Nun setzen wir $\frac{\mathrm d}{\mathrm d x} f(x) = 0$:
\begin{align*}
\frac{\mathrm d}{\mathrm d x} f(x) = 0 
    &\Leftrightarrow \log_e n \cdot \frac{\log_e x - 1}{\log_e^2 x} = 0 \\
    &\Leftrightarrow \log_e x - 1 = 0 \\
    &\Leftrightarrow x = e
\end{align*}
Und anschließend überprüfen wir, indem wir unser $x$ in $\frac{\mathrm
d^2}{\mathrm d x^2} f(x)$ einsetzen, ob es sich auch um das gewünschte Minimum
handelt:
\begin{align*}
\frac{\mathrm d^2}{\mathrm d x^2} f(e) &= \log_e n \cdot \frac{2 - \log_e e}{x
\cdot \log_e^3 e} \\
&= \log_e n \cdot \frac{2 - 1}{e \cdot 1^3} \\
&= \frac{\log_e n}{e}
\end{align*}
Daraus, dass das Ergebnis größer als 0 ist, folgt, dass es sich tatsächlich um
ein Minimum der Funktion handelt.

\item[(b)] Geplottet: \\
\begin{tikzpicture}
\draw[very thin, color=gray] (0,0) grid (9.9,9.9);
\draw[->] (0,0) -- (10.1,0) node[right] {$x$};
\draw[->] (0,0) -- (0,10.1) node[above] {$f(x)$};
\draw[domain=1.1:10, color=red] plot[id=log] function{x/log(x)};
\end{tikzpicture}

\end{enumerate}
\end{document}
