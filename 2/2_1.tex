\documentclass{article}
\usepackage[utf8]{inputenc}
\usepackage[ngerman]{babel}
\usepackage{amsmath}
\renewcommand{\labelitemi}{}

\begin{document}
\section{Aufgabe 1}
\begin{enumerate}
\item[(a)]
Ein $k$-närer Baum der Tiefe $l$ mit insgesamt $n$ Knoten hat 
\[\sum_{i=1}^l k^i - (\sum_{i=0}^l k^i - n)\] Kanten, dabei ist 
\begin{itemize}
\item $\sum_{i=0}^l k^i$ die Anzahl der Knoten des 
vollständigen Baumes, 
\item $\sum_{i=0}^l k^i - n$ die Anzahl der 
"`fehlenden Knoten"'
Dies entspricht $n-1$ Kanten in einem $k$-nären Baum der Tiefe l mit 
insgesamt $n$ Knoten. 
\end{itemize}

\item[(b)]
In einem Level liegen maximal $k^l$ Knoten.
\item[(c)]
$\sum_{i=0}^l k^i - b \cdot k$ \qquad mit b: Anzahl der Knoten 
ohne Kinder. \\
\item[(d)]
$\sum_{i=0}^l k^i = k^{l+1} - 1$
\end{enumerate}
\end{document}
