\documentclass{article}
\usepackage[utf8]{inputenc}
\usepackage[ngerman]{babel}
\usepackage{ansmath}

\begin{document}
\section{Aufgabe 2}
\begin{enumerate}
\item[(a)]
Die Laufzeiten der Algorithmen sind gleich, da sowohl ORDERx($leftChild$) 
($x \in \{1,2,3\} $) als auch ORDERx($rightChild$) $n$-mal, also an jedem 
Knoten aufgerufen wird, ebenso wie PRINT(v). Damit liegen alle 
Laufzeiten in $\Theta(n)$
\item[(b)]
\item[(c)]
\underline{Order1:}\\
\texttt{SRITEHGIEMOLWKAL}\\

\underline{Order2:}\\
\texttt{ETIHRIGESLOWMAKL}\\

\underline{Order3:}\\
\texttt{ETHIIEGRLWOALKMS}\\

\item[(d)]
\texttt{SVTNRIYUIE}
\item[(e)]
\texttt{WELIKEALGORITHMS}
\end{enumerate}
\end{document}
