\documentclass{article}
\usepackage[utf8]{inputenc}
\usepackage[ngerman]{babel}
\usepackage{amsmath}
\usepackage{algorithm}
\usepackage{algpseudocode}
\begin{document}

\section*{Aufgabe 3}

\textit{Lösung des Multiway-Cut Problems mittels lokaler Suche:}

Wir nehmen an, dass wie eine nicht-triviale Startlösung haben, zB indem 
man E' konstruiert, indem man alle Kanten aus E entfernt, die nicht zu 
einem Knoten der Teilmenge S hinführen.

Dann beginnen wir bei einem Knoten der Teilmenge S, nennen wir ihn a und 
prüfen, ob wir Kanten, die zu a hinführen, aus E' entfernen können, 
sodass immer noch keine zwei Knoten von S in derselben 
Zusammenhangskomponente von G-E' sind. 
Dann gehen wir alle Knoten von S derart durch.
 
\end{document}
