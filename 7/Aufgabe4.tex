\documentclass{article}
\usepackage[utf8]{inputenc}
\usepackage[ngerman]{babel}
\usepackage{amsmath}
\usepackage{algorithm}
\usepackage{algpseudocode}
\begin{document}

\section*{Aufgabe 4}

Um das Problem nach der Branch-and-Bound-Methode zu lösen benutzt man zunächst
einen beliebigen (Greedy-)Algorithmus um eine Anfangslösung zu finden. Von
dieser Anfangslösung, welche eine Vereinigung von Teilmengen, welche die
Zielmenge komplett überdecken, ist.

Nun gehen wir die Schritte wieder zurück und schauen an jedem Punkt, wie viele
Elemente der Zielmenge wir noch überdecken müssen und was für Mengen wir noch
dafür zur Verfügung haben. Als unterste Schranke dafür Berechnen wir die
geringste Anzahl an Teilmengen, so dass die Summe der Mächtigkeiten dieser
Mengen größer ist, als die Anzahl der Elemente, die wir noch überdecken müssen.
Dies ist wiederum ein Greedy-Algorithmus, welchen wir bereits in den
Präsenzaufgaben implementiert haben.

Die Untere Schranke, welche wir berechnen ist tatsächlich eine Untere Schranke,
da wir davon ausgehen, dass die Mengen disjunkt sind, wenn wir einfach nur ihre
Mächtigkeiten addieren. Sind die Mengen nicht disjunkt, so brauchen wir sogar
noch mehr Teilmengen, um die restlichen Elemente zu überdecken.


\end{document}
