\documentclass{article}
\usepackage[utf8]{inputenc}
\usepackage[ngerman]{babel}
\usepackage{amsmath}
\usepackage{algorithm}
\usepackage{algpseudocode}
\begin{document}

\section*{Aufgabe 1}

\begin{enumerate}
\item[(a)]
Gegeben: Die Mengen A und R. \\
Strategie: Wir nehmen das größte Element aus A und das größte Element aus R. 
Damit wird $a_i^{r_i}$ maximiert. Im nächsten Schritt nehmen wie das 
zweitgrößte Element aus A und aus R, und so weiter.
Dazu hofft man, dass entweder in A und R eine Null sind, oder nur in R, 
nicht aber nur eine in A, denn dann ist P immer 0.
 
\item[(b)]

Wenn P nicht maximal wär, dass bekäme man ein größeres P, indem man bis 
zu n Elemente anders zuordnet. Wenn man zwei Elemente in R tauscht, kann man 
die Differenz zwischen den Faktoren vermindern (zB ist $1 \cdot 9 < 5 
\cdot 5$), aber da Potenz stärker ist als Multiplikation, wird das Produkt 
insgesamt trotzdem kleiner.
Damit liefert die Stratgie stets den maximalen Wert von P.

\end{enumerate}

\end{document}
