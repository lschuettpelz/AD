\documentclass{article}
\usepackage[utf8]{inputenc}
\usepackage[ngerman]{babel}
\usepackage{amsmath}
\usepackage{algorithm}
\usepackage{algpseudocode}
\begin{document}

\section*{Aufgabe 1}

%\begin{algorithm}
%\begin{algorithmic}[5]

% Lauras Lösung
%\begin{verbatim}
%\# Sortiere die übergebe Folge in einen Stack "folge"
%
%\# Initialisiere die Stacks "best" "secondbest" und "reserve"
%
%\# Nimm das oberste Element vom Stack (letztes der Folge, und packe es in den 
%\# Stack best)
%
%best.push(stack.pop), boolean "bestchanged" = true
%
%\# Nimm das nächste Element vom Stack, falls es größer als das letzte Element 
%\# ist, dann wird kommt es in "'best"', sonst in "'secondbest"'
%
%if stack.peek > best.peek: 
%   secondbest.push(best.pop)
%   best.push(stack.pop)
%   bestchanged= true
%else
%   secondbest.push(stack.pop)
%   secondbestchanged = true
%   bestchanged = false 
%
%\# Wir brauchen nicht nur die Elemente, sondern auch deren Summen, diese wird 
%\# immer gleich mitberechnet (Stack erschien mir als schlaueste Form der 
%\# Speicherung)
%
%sum\_best = best.peek
%sub\_secondbest = secondbest.peek 
%
%\# Nun ist das nächste Element an der Reihe
%
%for elements in stack:
% 
%  a = folge.pop
%\# falls best nicht geändert wurde beim letzten durchlauf, addiere das Element
%\# zu best
%  if !bestchanged
%    best.pop(a)
%    bestchanged = true
%    secondbestchanged = false
%    sum\_best = sum\_best + a
%
%\# falls best geändert wurde, addiere das element zu secondbest
%
%  else 
%    secondbest.pop(a)
%    bestchanged = false
%    secondbestchanged = true
%    sum\_secondbest = sum\_secondbest + a
%
%\# falls secondbest nun größer ist als best, tausche die beiden
%
%    if sum\_secondbest > sum\_best
%      reserve = best
%      best = secondbest
%      secondbest = reserve
%      bestchanged = true
%      secondbestchanged = false
%
%\# falls das neue Element größer ist, als das, was letzte Runde in best 
%\# gelegt wurden, so lege das neue in best, entferne das alte, und das 
%\# vorherige best wird zu secondbest
%
%    else if a > best.peek
%      secondbest = best
%      best.pop
%      best.push(a)
%      secondbestchanged = false
%      bestchanged = true
%
% 
%\end{verbatim}     
%\end{algorithmic}
%\end{algorithm}

Wir schauen uns zunächst an, was für sehr kleine Listen als Ergebnis
herauskommt. Für eine Liste mit nur einem Element ist trivialer Weise genau das
eine Element die Teilliste. Bei zwei Elementen nimmt man das größere der beiden
Elemente. Ab dem dritten Element wird es bereits etwas schwieriger. Hier nimmt
man entweder das erste Element und das letzte oder das vorletzte Element. 

Bei vier Elementen gibt es wieder die zwei Fälle, dass man entweder das letzte
Element nimmt und das vorletzte nicht oder umgekehrt, aber der Teil davor spielt
auch noch eine Rolle. Wenn man genau hinschaut muss man hier jedoch auch nur
zwei Gewichte berechnen. Einmal das Gewicht (oder die Summe) der Teilliste, wenn
man das letzte Element herauslässt oder das Gewicht vom letzten Element zusammen
mit der Summe der maximalen Subliste (unter den Bedigungen der Aufgabe) der
Teilliste ohne die letzten beiden Elemente.

Dies lässt sich rekursiv formulieren:
\begin{verbatim}
be A[1...n] the input list
maxWeight(i)
    return max( maxWeight(i-1) , maxWeight(i-2) + A[i] )
\end{verbatim}

Bottom Up lässt sich das Problem durch folgenden Algorithmus in \(O(n)\) lösen:
\begin{verbatim}
be A[1...n] the input list

if only one Element: return Element
if two Elements: return bigger Element

Weights[1] = A[1]
Weights[2] = max(A[1], A[2])

Function(A)
    for i from 3 up to n:
    if (Weights[i-2] + A[i]) > Weights[i-1]
        Weights[i] = Weights[i-2] + A[i]
    else
        Weights[i] = Weights[i-1]
    end for
    return Weights[n]
\end{verbatim}

Angepasst daran, dass das Ergebnis - also die Liste, welche die Summe ergibt -
gleich mit ausgegeben wird, ist das Folgende:

\begin{verbatim}
be A[1...n] the input list

if only one Element: return Element
if two Elements: return bigger Element

Weights[1] = A[1]
Weights[2] = max(A[1], A[2])
Sequence[1] contains A[1]
Sequence[2] contains max(A[1], A[2])

Function(A)
    for i from 3 up to n:
    if (Weights[i-2] + A[i]) > Weights[i-1]
        Weights[i] = Weights[i-2] + A[i]
	Sequence[i] = Sequence[i-2] with A[i]
    else
        Weights[i] = Weights[i-1]
	Sequence[i] = Sequence[i-1]
    end for
    return Weights[n]
\end{verbatim}

Da je nur die beiden letzten Elemente von \texttt{Weights} und \texttt{Sequence}
benutzt werden, kann man sich auch das Array sparen und stattdessen mit je zwei
Variablen arbeiten, was den Speicherbedarf des Algorithmus von \(O(n)\) auf
\(O(1)\) reduziert.

Die Laufzeit liegt genau deshalb in \(O(n)\), weil die \texttt{for}-Schleife
ungefähr n mal ausgeführt wird. Da es keine anderen Bedingungen für die
Termination oder Rekursionen gibt, bestimmt nur die Schleife die Laufzeit.

\end{document}
