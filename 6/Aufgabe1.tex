\documentclass{article}
\usepackage[utf8]{inputenc}
\usepackage[ngerman]{babel}
\usepackage{amsmath}
\usepackage{algorithm}
\usepackage{algpseudocode}
\begin{document}

\section*{Aufgabe 1}

\begin{algorithm}
\begin{algorithmic}[5]

\Function{$FloydWarshall*$}{$W$}
\State $ n \gets \text{number of vertices}$ \\
\State $D^{(0)} \gets W$ 
\For {$s = 1, \dots, n$}
\For {$t = 1, \dots, n$}
\If{$d_0(s,t)=\infty$}
\State $m(s,t) = NIL$
\Else  
\State $m(s,t) \gets s$
\State $\text{ \#Initialize predecessor matrix m}$
\EndIf
\EndFor
\EndFor
\For {$k = 1, \dots, n$}
\For {$s = 1, \dots, n$}
\For {$t = 1, \dots, n$}
\State $d_k(s,t) \gets min\{d_{k-1}(s,t), d_{k-1}(s,k) + d_{k-1}(k,t)\}$\\
\If{$d_k(s,t) < d_{k-1}(s,t)$}
\State $m(s,t) \gets k$ \\
\State $\text{\# If the path gets shorter, the predecessor must have changed,}$
\State $\text{\# and since we just tried out k, it must be k}$
\EndIf
\EndFor
\EndFor
\EndFor


\EndFunction

\end{algorithmic}
\end{algorithm}

Die Matrix M wird so initialisiert, dass falls von s nach t ein Pfad 
besteht, s als Vorgänger von t angegeben wird, und wenn nicht, es noch 
keinen Vorgänger für t gibt. 
Läuft nun der Algorithmus und wird durch Hinzunehmen des Knotens k ein 
kürzerer Pfad gefunden, wird k als neuer Vorgänger eingesetzt. 

Am Ende stehen in der Matrix M jeweils die Vorgänger, bei denen ein 
kürzester Weg gefunden wurde, wenn also bei dem k-ten Durchlauf k als 
neuer Vorgänger für t von s aus festgelegt wurde, kann man in der 
Matrix schauen, welcher Vorgänger für k eingetragen ist, auf dem 
kürzesten Weg von s nach k. Damit kann man aus der Matrix den kürzesten 
Pfad ablesen. 

Das könnte man jetzt über Induktion zeigen, aber soviel Zeit habe 
ich nicht. 

\end{document}
