\documentclass{article}
\usepackage[utf8]{inputenc}
\usepackage[ngerman]{babel}
\usepackage{amsmath}
\usepackage{algorithm}
\usepackage{algpseudocode}
\begin{document}

\section*{Aufgabe 5}

\begin{enumerate} 
\item[(a)] 

Gelte die Aussage nicht. Dann gäbe es einen Graphen G mit einen 
Schnitt (S, V \textbackslash S) von G mit einer eindeutig 
bestimmten Schnittkante, so dass G keinen eindeutigen minimalen 
Spannbaum besitzt. Sei der minimale Spannbaum T nicht eindeutig, 
dann gibt es die minimalen Spannbäume T und T'. T und T' 
unterscheiden sich damit in mindestens einer Kante, e in T und e' 
in T'. 

OBdA können wir annehmen, dass wir einen Schnitt so durch den 
Graphen legen können, dass e und e' Schnittkanten sind. Sind noch 
weitere Kanten im Schnitt, so ist ihr Gewicht höher (dann sind sie 
definitiv nicht die minimale Schnittkante), oder niedriger (dann 
ist diese Kante in T und T', e und e' aber nicht), es kann nicht 
gleich sein, da es nach Voraussetzung eindeutige minimale 
Schnittkanten gibt.

Nach eben dieser Voraussetzung gibt es aber nur eine minimale 
Schnittkante, e oder e', und damit muss e' > e gelten (oder anders 
herum). Damit ist T' aber auch kein minimaler Spannbaum mehr.
Also muss e = e' und T = T' gelten, damit ist der minimalle Spannbaum eindeutig bestimmt.


\item[(b)]

Die Umkehrung der Aussage: Wenn G einen eindeutig bestimmten 
minimalen Spannbaum hat, dann existiert für jeden Schnitt von G 
eine eindeutig bestimmte Schnittkante mit minimalem Gewicht.

Man nehme einen Graphen, der aus drei Knoten und drei Kanten 
besteht, die das Gewicht 1, 1 und 2 haben. Der minimale Spannbaum 
ist eindeutig bestimmt. Aber legt man einen Schnitt so durch G, 
dass beide Kanten mit Kantengewicht 1 durchschnitten werden, so 
ist die Schnittkante nicht eindeutig.

\end{enumerate}

\end{document}
