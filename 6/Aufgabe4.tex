\documentclass{article}
\usepackage[utf8]{inputenc}
\usepackage[ngerman]{babel}
\usepackage{amsmath}
\usepackage{algorithm}
\usepackage{algpseudocode}
\begin{document}

\section*{Aufgabe 4}

\begin{enumerate}
\item[(a)]
Annahme: T ist kein minimaler Spannbaum im veränderten Graphen, also 
hat sich T zu T' = (V, E'') (der neue minimale Spannbaum) geändert. 
Da nur die Kante $e \in E \backslash E$' verändert wurde, muss e in E''
liegen, e also im neuen minimalen Spannbaum sein.

Da e erhöht wurde, wäre dann aber e auch schon in E', also im 
vorherigen minimalen Spannbaum gewesen. Dies ist aber nicht 
gegeben, damit muss T auch im veränderten Graphen der minimale 
Spannbaum sein.
  
\item[(b)]
Zuerst prüfen wir, ob e in E' liegt. Wenn ja, dann bleibt T minimaler 
Spannbaum. 
Wenn nein, dann muss geprüft werden, ob sich der minimale Spannbaum 
verändert hat.  
Dazu kann man sich die beiden Knoten heraussuchen, die e verbindet, 
alle von diesen aus- und eingehenden Kanten aus T herausnehmen, und dann
auf diesen Knoten Kruskal oder Prims Algorithmus anwenden, um 
herauszufinden, welche dieser Kanten in einen Spannbaum gehören, sodass 
dieser minimal und ohne Zyklen ist.

Das Verfahren ist korrekt. Wenn $e \in E$', dann ist T weiterhin 
minimaler Spannbaum, mit einem verringerten Kantengewicht. Wenn $e 
\notin E$', dann muss geprüft werden, ob T sich dadurch verändert. 
T kann sich nur an Kanten verndern, die direkt in der Umgebung von 
e liegen. Den Rest der Kanten importieren wir, und prüfen nur, wie 
die "`offenen Knoten"' am minimalsten eingebunden werden können. 
Kruskal und Prim sind laut Vorlesung korrekt, damit ist das 
Verfahren korrekt.
 
\end{enumerate}

\end{document}
