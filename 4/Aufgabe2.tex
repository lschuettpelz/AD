\documentclass{article}

\usepackage[ngerman]{babel}
\usepackage[utf8]{inputenc}
\usepackage{amsmath}
\usepackage{amssymb}
\usepackage{algorithmicx}
\usepackage{enumitem}

\begin{document}
\section*{Aufgabe 2}
\begin{enumerate}[label=(\alph*)]
\item Diese Aussage ist wahr. Man nehme einen Pfad \(P \in V^*\) vom Knoten
\(v_0\) zum Knoten \(v_n\) mit Knoten \(v_i \in V\), in dem mehrere Knoten
mehrfach vorkommen können. Man nimmt sich zwei dieser Knoten heraus und nennt
diese \(v_k\), so dass \(P = \left( v_0, v_1, \dots , v_k , \dots , v_k , \dots
, v_{n-1}, v_n\right)\) gilt. Dieser Pfad lässt sich in drei Pfade \(P'_i$ mit
$i \in \{0,1,2\}\) unterteilen, wobei folgendes gilt:
\[
	P_0 = \left( v_0 , \dots , v_k \right), \quad
	P_1 = \left( v_k , \dots , v_k \right), \quad
	P_2 = \left( v_k , \dots , v_n \right)
\]

Aus \(P_0\) und \(P_2\) kann man nun aber einen Pfad \(P'\) mit \(P' = \left(
v_0 , \dots , v_k , \dots , v_n \right) \) konstruieren, der von \(v_0\) nach
\(v_n\) führt, aber nur einmal den Knoten \(v_k\) passiert. Dieses Verfahren
kann so lange wiederholt werden, bis kein Knoten mehr doppelt vorkommt.

\item Im worst-case besteht der zusammenhängende Graph nur aus einem einzelnen
Pfad, in dem jeder Knoten nur einmal vorkommt. Dann ist die maximale Distanz
zwischen den beiden Endknoten die länge des Pfades, welche \(n-1\) ist. Jede
weitere Kante, die hinzugefügt wird, erzeugt eine Schleife im Graphen. Da eine
Schleife den kürzesten Pfad nicht länger machen kann (siehe Aufgabenteil (a)),
wird der längste kürzeste Pfad dadurch nicht länger als \(n-a\).

\item Ein Baum ist per Definition schleifenfrei. Daher können je zwei Knoten
auch nur durch einen Pfad miteinander verbunden werden.

\item Ja, per Definition.

\item Nein. Die Summe aller Knotengrade in einem ungerichteten Graphen mit \(m\)
Kanten ist genau \(2m\). Dies liegt daran, dass jede Kante genau aus zwei Knoten
herausläuft und der Grad durch die Anzahl der verlassenden Kanten eines Knotens
definiert ist.

\item Da, da die Länge des Pfades von einem Blatt zum Wurzelknoten maximal
\(\lceil \log_k n \rceil\) ist. Da der Pfad zwischen zwei Blättern in einem Baum
immer über den Wurzelknoten läuft, kann diese Distanz maximal die doppelte
maximale Distanz eines Blattes zum Wurzelknoten sein.

\item Ja, man kann die Zahl wie folgt herleiten:\\
Der erste Knoten hat zu maximal \(n-1\) anderen Knoten eine Verbindung. Der
nächste Knoten hat nur noch zu \(n-2\) Knoten eine Verbindung, die zuvor nicht
existierte. Das geht so lange weiter, bis der letzte Knoten lediglich eine
weitere Kante hinzufügt. Schon der kleine Gauß wusste in der Grundschule, dass
dies \(\frac{(n-1)(n-1+1)}{2} = \frac{(n-1)(n)}{2}\) ergibt (Gauß'sche
Summenformel). Daher stimmt die Formel.

\item Diese Aussage stimmt nicht, da ein vollständiger Graph diese Bedingung
erfüllt.


\end{enumerate}

\end{document}
