\documentclass{scrartcl}

\usepackage[utf8]{inputenc}
\usepackage[ngerman]{babel}
\usepackage{amssymb, amsmath}

\begin{document} 

\section{Aufgabe 3}

\begin{enumerate} 
\item[(a)] 
\underline{Induktionsanfang}: \\
Sei $k=0$. Dann ist $A^0=I$, I die Einheitsmatrix, und \\
$A^0[i,j]= \begin{cases} &1 \text{ für } i=j, \\
			&0 \text{ sonst } \end{cases}$ \\ 
da jeder Knoten eine Kante der Länge 0 zu sich selbst hat.

\underline{Induktionsannahme}: \\
Sei $A^k[i,j]$ genau die Anzahl verschiedener Pfade der Länge $k$, 
die in $G$ von $i$ nach $j$ führen. 

\underline{Induktionsschritt}: \\
\begin{align*} 
A^{k+1}&=A^k \cdot A \\ 
A^{k+1}[i,j]&= \sum_{m=1}^n a_{im} \cdot b_{mj} \\
\text{ dabei ist } a_{im} \text{ Element der Matrix } A^k 
\text{ und } b_{mj} \text{ Element der Matrix } A. 
\end{align*} 
Dann steht nach Induktionsannahme in $a_{im}$ genau die Anzahl 
verschiedener Pfade der Länge $k$, die in $G$ von $i$ nach $m$ führen.
In $b_{kj}$ steht entweder eine 1 oder eine 0, je nachdem, ob von $m$ ein 
Pfad nach j führt. Durch die Multiplikation werden alle Pfade 
aussortiert, die zwar von $i$ nach $m$, aber nicht weiter nach $j$ führen 
(Multiplikation mit 0 = 0), und diejenigen Pfade dazu addiert, die von 
$i$ nach $m$ und dann weiter nach $j$ führen. 

Damit sind in $A^{k+1}[i,j]$ genau die Einträge, die mit Länge $k$ von 
$i$ nach $m$ führen, und dann mit einem Schritt nach $j$, also genau 
die Anzahl verschiedener Pfade, die von $i$ nach $j$ mit Länge $k+1$.


\item[(b)] \item[(c)] \end{enumerate}



\end{document}
