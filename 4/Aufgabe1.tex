\documentclass{scrartcl}

\usepackage[utf8]{inputenc}
\usepackage[ngerman]{babel}
\usepackage{amssymb, amsmath}

\begin{document} 
\section{Aufgabe 1} 
\begin{enumerate}
\item[(a)]
Wir testen mit dem ersten 
Glas jeweils in $m^2$ Schritten, d.h. Sprosse (1, 4, 9, 16, 25, . 
. . ). Dies ergibt einen Aufwand von $\sqrt{n}$. Wenn auf einer Stufe das 
erste Glas bricht, gehen wir zur letzten getesteten Stufe zurück 
und testen von dort schrittweise jede Stufe, bis das neue Glas 
ebenfalls zerbricht. Der Gesammtaufwand beträgt also $\sqrt{n} + (\sqrt{n} - \sqrt{n - 1}) = \sqrt{n}(1 - \sqrt{n - 1}) \in o(n)$.

\item[(b)] 
Bei der linearen Suche zerstören wir genau ein Glass. 
Bei der BinarySearch zerstören wir im schlimmsten Fall log n 
Gläser zerstören. Wir beherrschen dieses Trade-off, indem wir $l-1$ 
mal Binary-Search durchführen, und dann in dem Bereich, den wir 
eingrenzen konnten, lineare Suche durchführen.
\end{enumerate}

\end{document}
