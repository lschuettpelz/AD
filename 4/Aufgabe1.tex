\documentclass{scrartcl}

\usepackage[utf8]{inputenc}
\usepackage[ngerman]{babel}
\usepackage{amssymb}{amsmath}

\begin{document} \section{Aufgabe 1} Bei der linearen Suche 
zerstören wir genau ein Glass. Bei der BinarySearch zerstören wir 
im schlimmsten Fall log n Gläser zerstören. Wir beherrschen dieses 
Trade-off, indem wir l-1 mal Binary-Search durchführen, und dann 
in dem Bereich, den wir eingrenzen konnten, lineare Suche 
durchführen. 

Der schlimmste Fall, wie oft wir ein Glas fallen lassen müssen, 
ist weiterhin, wenn wir sagen l=1, wir möchten also nur ein Glas 
kaputt machen und machen lineare Suche. Bei l+1 machen wir Binary 
Search, im schlimmsten Fall zerbricht das Glas sofort und wir 
machen lineare Suche in der unteren Hälfte der Leiter. 

Dies Prinzip wird auch bei höheren l so fortgesetzt, wenn wir bei 
l genau m Gläser fallen lassen müssen, dann können wir bei l+1 
genau einmal öfter Binary Search durchführen und die Anzahl der 
Gläser, die man fallen lassen muss halbieren($\frac{m}{2}$).


\end{document}
