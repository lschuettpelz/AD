\documentclass{article}

\usepackage[ngerman]{babel}
\usepackage{amsmath}
\usepackage{amssymb}
\usepackage{algorithm}
\usepackage{algorithmic}

\begin{document}

\section*{Aufgabe4}
\begin{enumerate}
\item[(a)] Folgendes Verfahren ist eine Adaption des Bellman-Ford-Algorithmus.
Der Algorithmus wurde so angepasst, dass er statt die Gewichte zu addieren,
diese multipliziert, da es sich hier bei den Kantengewichten um
Wahrscheinlichkeiten handelt.

\begin{algorithm}
\caption{InitializeSingleSource*(G,s)}
\begin{algorithmic}
\FORALL{\(v \in V\)}
\STATE \(v.dist \leftarrow 0\)
\STATE \(v.\pi \leftarrow NIL\)
\ENDFOR
\STATE \(s.dist \leftarrow 1\)
\end{algorithmic}
\end{algorithm}

\begin{algorithm}
\caption{relax*(u,v)}
\begin{algorithmic}
\IF{\( v.dist < u.dist \cdot w(u,v) \)}
\STATE \( v.dist = u.dist \cdot w(u,v) \)
\STATE \( v.\pi = u \)
\ENDIF
\end{algorithmic}
\end{algorithm}

\begin{algorithm}
\caption{BellmanFord*(G,s)}
\begin{algorithmic}
\STATE InitializeSingleSource*(G,s)
\FOR{\( i = 1,\dots,\left|V\right| - 1 \)}
\FORALL{\( edges (u,v) \in E \)}
\STATE Relax*(u,v)
\ENDFOR
\ENDFOR
\FORALL{\( edges (u,v) \in E \)}
\IF{ \(v.dist < u.dist \cdot w(u,v) \)}
\RETURN false
\ENDIF
\ENDFOR
\RETURN true

\end{algorithmic}
\end{algorithm}

\end{enumerate}

\end{document}
