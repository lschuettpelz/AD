\documentclass{article}

\usepackage[ngerman]{babel}
\usepackage{amsmath}
\usepackage{algorithmic}

\begin{document}

\section*{Aufgabe1}
Folgender Algorithmus bekommt einen topologisch sortierten Graphen, der
keine Schleifen enth"alt, "ubergeben und wendet dann auf dem Graphen direkt
die Relax-Operationen an.

\begin{algorithmic}
\STATE NodeQueue \(\leftarrow\) \(\left[StartNode\right]\)
\WHILE{NodeQueue not empty}
    \STATE \(u := DEQUEUE(NodeQueue)\)
    \FORALL{\(v \in Adj(u)\)}
	\STATE \(Relax(u,v)\)
	\STATE \(ENQUEUE(NodeQueue, v)\)
    \ENDFOR
\ENDWHILE
\end{algorithmic}

Da die Gesamtanzahl der Kanten eines Graphen der Summe aller Kanten, die in
die einzelnen Knoten hineinf"uhren entspricht, wird jeder Knoten genau so oft
in die Queue eingef"ugt. Die While-Schleife wird, da jedes mal maximal ein
Knoten aus der Queue herausgenommen wird, daher f"ur jede Kante in dem
Graphen genau ein Mal aufgerufen. Zus"atzlich wird die Schleife einmal am
Anfang f"ur den Startknoten aufgerufen.

Dieser Algorithmus entspricht der Breitensuche. Er markiert jedoch keine
Knoten, da es f"ur unseren Schleifenfreien, gerichteten Graphen m"oglich
ist, einen Knoten mehrfach zu besuchen und dennnoch zu einer Termination des
Algorithmus kommt. Dies ist wie weiter oben beschrieben genau nach
\(\left|E\right|\) Schritten der Fall.

\end{document}
